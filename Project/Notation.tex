%%\appendix
%%\appendixpage
%%\addappheadtotoc

%% \section{Notation Reference}

%% We use consistent variable names throughout this paper.
%% For convenience, we collect them here.  
%% \begin{figure}
%% \caption{The notation used in this project}
%% \label{fig:notation}
%% \begin{tabular}{ll}
%%   $s$ & a tweet\\
%%   $u$ & a user\\
%%   $t$ & a topic\\
%%   $p$ & a time\\
%%   $F_u$ & number of followers of user $u$\\
%%   $u\to u'$ & indicator variable of user $u$ follows user $u'$\\
%%   $R_{u,u'}$ & number of times $u$ retweets $u'$\\
%%   $R_{u}$ & number of times $u$ is retweeted; $\sum_{u'}R_{u',u}$\\
%%   $\mathcal{R}_s^{(p)}$ & set of users who retweet $s$.\\
%%   $R_s^{(p)}$ & number of users who retweet $s$.\\
%%   $TR_{u,t}$ & \texttt{TwitterRank} of user $u$ for topic $t$\\
%%   $RR_{u,t}$ & \texttt{RetweetRank} of user $u$ for topic $t$\\
%%   $RR_u$ & total \texttt{RetweetRank} of user $u$; $\sum_tRR_{u,t}$ \\
%%   $RR_{s,t}^{(p)}$ & sum of $RR_{u,t}$ over $u\in\mathcal{R}_{s}^{(p)}$\\
%%   $RR_{s}^{(p)}$ & sum of $RR_u$ over $u\in\mathcal{R}_{s}^{(p)}$\\
%%   $\alpha_s$ & total alpha of $s$; the quality of tweet $s$\\
%%   $\phi_{u,t}$ & probability of $u$ to retweet perfect tweet of topic $t$\\
%%   $\tau_s$ & topic of tweet $s$\\
%%   $\alpha_s^{(p)}$ & the partial alpha computed at time $p$\\
%%   $\phi_s^{(p)}$ & sum of $\phi_{u,\tau_s}$ over $u\in\mathcal{R}_s^{(p)}$\\
%% \end{tabular}
%% \end{figure}
