\section{\texttt{TwitterRank} and \texttt{RetweetRank}}

\subsection{Overview The \texttt{TwitterRank} Algorithm}

Our first high powered feature is a modified \texttt{TwitterRank}, as in \cite{Weng:2010:TFT:1718487.1718520}.
The original \texttt{TwitterRank} works as follows.  
First use LDA\footnote{We used the package \texttt{Mr.LDA} to compute LDA through Variational Inference on Map Reduce \cite{conf/www/ZhaiBAA12}.} to compute topic scores between each pair of users $u$ and $u'$ and each topic $t$, denoted $\mbox{sim}_t(u,u')$.
Then, for each topic $t$, form the weighted graph where each node is a user there is a directed edge between $u$ and $u'$ if $u$ follows $u'$.
The graph weights, $w^{TR}_t$, are given by
\begin{equation}\label{eq:treq}
w^{TR}_t(u,u') = \frac{T_{u'}}{\sum_{a:u\to a}T_{a}}\cdot \mbox{sim}_t(u,u')
\end{equation}
where $T_{a}$ is the number of tweets posted by user $a$ and $u\to a$ is the binary variable indicating $u$ follows $a$.  
For each $t$, we run weighted \texttt{GraphRank}, where with teleportation probabilities given by normalized topic strength of topic $t$, $\gamma_{u,t}$.   
The resulting rank is the $t$-specific \texttt{TwitterRank}, denoted $TR_{u,t}$.  

\begin{algorithm}
  \caption{Topic Specific TwitterRank}
  \label{alg:tralg}

  \begin{algorithmic}
    \State Use LDA to compute $\gamma_{u,t}$ and $\mbox{sim}_t(u,u')$ for each topic $t$ and user pair $u,u'$.  
    \For {$t\in$ Topics}
    \State Initialize $TR_{u,t}$ to 1 for each user $u$.
    \While {Not converged}
    \For {$u\in$ Users}
    \State $TR_{u,t}\gets \gamma_{u,t}(1-\beta) + \beta w^{TR}_t(u',u)\cdot TR_{u',t}$
    \EndFor
    \EndWhile
    \EndFor
  \end{algorithmic}
\end{algorithm}


Intuitively, the edge weight $w_t(u,u')$ is what fraction of $u$'s twitter feed $u'$ takes up times their topic similarity.
That is, user $u$, when randomly viewing his twitter feed, will randomly pick someone, but weighted such that they will topic pick a user who's topic interests them.  
Considering this as a random surfer is then slightly far-fetched, since the user does not then get to see his friend's feed, but we do it anyway.  
The idea is that users with more influence in a specific topic will dominate the feeds of users with high influence in that same topic, and this captures that notion exactly.

\subsection{The \texttt{RetweetRank} Algorithm}

The \texttt{RetweetRank} gives a high powered feature of the user which measures their influence in terms of accumulating retweets.  
The idea, if you are retweeted by someone who's tweets have a high retweet rank, then you ought to have a powerful propensity for making your tweets visible as well.
In the fable of the random surfer, the tweet itself can be though of as surfing.  

We modify this algorithm for our purposes in two slightly different ways.
First, we consider a slightly different graph; the retweet graph, where each user is a node and there is an edge between $u$ and $u'$ if $u$ has ever retweeted $u'$.
This is both stronger and weaker than the follower graph; since one user can follow another without ever retweeting them, and one user can retweet another without following them.
The latter happens in practice quite often, as a ``retweet of a retweet'' is indistinguishable from a single retweet.
If $u$ retweets $v$ and $w$ sees $u$'s retweet and retweets it, it is only recorded that $w$ retweeted $v$, and this can and does happen even if $w$ is not following $v$ at all!

Second, we modify the edge weights as follows.
\begin{equation}\label{eq:rreq}
w^{RR}_t(u,u') = \frac{R_{u,u'}T_{u'}}{\sum_a R_{u,a}T_{a}}\cdot \mbox{sim}_t(u,u')
\end{equation}
where $R_{u,u'}$ is the number of times $u$ has retweeted $u'$ and where the sum over $a$ is over all users.  

To compute the \texttt{RetweetRank}, $RR_{u,t}$, use Algorithm~\ref{alg:tralg} with equation~\eqref{eq:treq} replaced with equation~\eqref{eq:rreq} in the update step, with the sum now over all users who have retweeted $u$, not just followers.  

\subsection{Tweet Features from Retweet Rank}

Given a brand new tweet $s$ from user $u$, the $RR_u^t$ give a strong and immidiate first guess at the eventual popularity of the tweet.  If, further, we wait until time $p$ after a tweet is posted, we can compute tweet features
\[RR_{s,t}^{(p)} = \sum_{u\in\mathcal{R}_s^{(p)}}RR_{u,t}\]
where $\mathcal{R}_s^{(p)}$ is the set of users who have retweeted $s$ within a time $p$ after $s$ was posted.  
We compute these for a few values of $p$.
Finally, if we wish to remove features we can compute overall \texttt{RetweetRank},
\[RR_{u} =\sum_t RR_{u,t}\]
\[RR_{s}^{(p)} \sum_t RR_{s,t}^{(p)}\]
Note that it was still useful to have topics when computing the ranks originally, as this allows for a much richer space of models, even if we only ever use the aggregate over topics.    

